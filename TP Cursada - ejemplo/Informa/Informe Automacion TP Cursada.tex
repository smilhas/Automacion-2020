%% LyX 2.2.1 created this file.  For more info, see http://www.lyx.org/.
%% Do not edit unless you really know what you are doing.
\documentclass[spanish]{article}
\usepackage[T1]{fontenc}
\usepackage[latin9]{inputenc}
\usepackage{geometry}
\geometry{verbose,tmargin=2cm,bmargin=2cm,lmargin=2cm,rmargin=2cm}
\usepackage{xcolor}
\usepackage{float}
\usepackage{endnotes}
\usepackage{graphicx}

\makeatletter
%%%%%%%%%%%%%%%%%%%%%%%%%%%%%% Textclass specific LaTeX commands.
 \let\footnote=\endnote

\makeatother

\usepackage{babel}
\addto\shorthandsspanish{\spanishdeactivate{~<>.}}

\begin{document}
\begin{center}
\includegraphics[width=10cm]{\string"../../../Diseno de Equipos Electronicos/_Proyecto/Informes/_INFORME/Imagenes/ITBA logo\string".eps}
\par\end{center}

\begin{center}
\textbf{\Huge{}Automaci�n Industrial}\\
\textbf{\Huge{}}\\
\textbf{\Huge{}}\\
\textbf{\Huge{}}\\
\textbf{\Huge{}}\\
\textbf{\Huge{}}\\
\textbf{\Huge{}}\\
\par\end{center}{\Huge \par}

\begin{center}
\textbf{\textcolor{gray}{\Huge{}Trabajo Pr�ctico de Cursada}}\\
\textbf{\textcolor{gray}{\Huge{}}}\\
\textbf{\textcolor{gray}{\Huge{}}}\\
\textbf{\textcolor{gray}{\Huge{}}}\\
\textbf{\textcolor{gray}{\Huge{}}}\\
\textbf{\textcolor{gray}{\Huge{}}}\\
\textbf{\textcolor{gray}{\Huge{}}}\\
\textbf{\textcolor{gray}{\Huge{}}}\\
\textbf{\textcolor{gray}{\Huge{}}}\\
\textbf{\textcolor{gray}{\Huge{}}}\\
\par\end{center}{\Huge \par}

\begin{flushleft}
\textbf{\Large{}Autores:}
\par\end{flushleft}{\Large \par}

\begin{center}
{\Large{}Franco, Tom�s Mario - 53.777}
\par\end{center}{\Large \par}

\begin{center}
{\Large{}Garc�a Eleisequi, Santiago - 50.089}
\par\end{center}{\Large \par}

\begin{flushleft}
\textbf{\Large{}Tutores}{\Large{}:}
\par\end{flushleft}{\Large \par}

\begin{center}
{\Large{}Ghersin, Alejandro Sim�n}
\par\end{center}{\Large \par}

\begin{center}
{\Large{}Arias, Rodolfo Enrique}
\par\end{center}{\Large \par}

\begin{center}
\pagebreak{}
\par\end{center}

\tableofcontents{}

\listoffigures

\listoftables

\newpage{}

\section*{Introducci�n}

En el presente trabajo se estudi� el control tanto de fuerzas como
de posici�n de un brazo del tipo RR. El mimso est� limitado por una
pared que intersecta al $eje\ XY$ en los puntos $P_{1}=(2,0)$ y
$P_{2}=(0,2)$ ubicada en un plano. A continuaci�n se muestra una
esquematizaci�n del sistema a estudiar.

\begin{figure}[H]
\begin{centering}
\includegraphics[width=10cm]{\string"Imagenes/Esquema de trabajo\string".eps}
\par\end{centering}
\caption{Esquema de trabajo.}

\end{figure}


\section{Control de posici�n no lineal de un manipulador rob�tico}

Se desarrolla a continuaci�n un controlador cartesiano de posici�n
que hace que el efector final del manipulador se desplace desde el
punto $P_{inicial}=(1,-1)$ hasta el punto $P_{final}=(1,1)$.

\subsection{An�lisis inicial}

A continuaci�n se observa en la figura \ref{fig: E1 Joint vs time}
la gr�fica de como var�an las cordenadas joint respecto al tiempo.
En rojo se presenta la variable $q_{1}$ que representa el �ngulo
$\theta_{1}$; y en negro $q_{2}$ que representa el �ngulo $\theta_{2}$.

\begin{figure}[H]
\begin{centering}
\includegraphics[width=8cm]{\string"Imagenes/Coordenadas Joint\string".eps}
\par\end{centering}
\caption{Gr�ficas de las coordenadas joint en el tiempo.\label{fig: E1 Joint vs time}}

\end{figure}

Se presenta en el gr�fico de la figura \ref{fig: E1 Coord XY respecto a t}
las coordenadas cartesianas $x$ e $y$de la posici�n del efector
final. En rojo se observa la posici�n deseada $\vec{X}$y en azul
se contrasta la posici�n real del efector final. Se observa que las
trayectorias son similares; si bien en el gr�fico de la coordenada
$x$ hay una gran diferencia respecto de las gr�ficas es porque se
realiz� un zoom para mostrar dicha diferencia.

\begin{figure}[H]
\begin{centering}
\includegraphics[width=8cm]{\string"Imagenes/Real Vs Deseada XYt\string".eps}
\par\end{centering}
\caption{Gr�ficas de las coordenadas cartesianas respecto al tiempo.\label{fig: E1 Coord XY respecto a t}}
\end{figure}

Se presenta a continuaci�n una manera de observar la trayectoria,
como la coordenada $y$ respecto de $x$. Tambi�n se presenta aqu�
en rojo la trayectoria deseada o ideal y en azul la trayectoria real.

\begin{figure}[H]
\begin{centering}
\includegraphics[width=8cm]{\string"Imagenes/Real Vs Deseada XY\string".eps}
\par\end{centering}
\caption{Gr�ficas de las coordenadas cartesianas X respecto de Y.}
\end{figure}


\subsection{An�lisis con 80\% error}

\section{Control de fuerza no lineal de un manipulador rob�tico}

\section{Control h�brido no lineal de un manipulador rob�tico}
\end{document}
